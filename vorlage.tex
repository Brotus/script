\documentclass[a4paper,12pt,fleqn]{scrartcl}
 
%**** Benutzte Packages

% utf8 Zeichen verwendbar
\usepackage[utf8x]{inputenc}
\usepackage[T1]{fontenc}
\usepackage[ngerman]{babel}
\usepackage{amsmath}
\usepackage{amssymb}
\usepackage{amsfonts}
\usepackage{amsthm}
\usepackage{tikz}
\usepackage{mathtools}
%http://www.guntherkrauss.de/computer/tex/diagramme.html
\usepackage[arrow, matrix, curve]{xy}


%**** Selbst definierte Befehle

% Abkürzungen
\newcommand{\gdw}{\Leftrightarrow}
\newcommand{\N}{\mathbb{N}}
\newcommand{\Z}{\mathbb{Z}}
\newcommand{\Q}{\mathbb{Q}}
\newcommand{\R}{\mathbb{R}}
\newcommand{\C}{\mathbb{C}}
\newcommand{\F}{\mathbb{F}}
\newcommand{\impl}{\Rightarrow}
\newcommand{\la}{\lambda}
\newcommand{\al}{\alpha}
\newcommand{\SeiV}{Sei $V$ ein $K$-Vektorraum }
\newcommand{\summeA}{\sum_{i\in I}}
\newcommand{\summeB}{\sum_{i=1}^n}
\newcommand{\ol}[1]{\overline{#1}}
\newcommand{\norm}[1]{\|#1\|}
\newcommand{\fa}[1]{\mathop{\forall}\limits_{#1}}
\newcommand{\ex}[1]{\mathop{\exists}\limits_{#1}}
\newcommand{\sk}{\mbox{}\\*}

%rowcounter zurücksetzen
\newcommand{\resetRows}{\setcounter{equation}{0}}

%**** Theoreme

\theoremstyle{definition}
\newtheorem*{definition}{Definition}
\newtheorem*{satz}{Satz}
\newtheorem*{theorem}{Theorem}

\theoremstyle{plain}
\newtheorem*{lemma}{Lemma}
\newtheorem*{korollar}{Korollar}

\theoremstyle{remark}
\newtheorem*{bemerkung}{Bemerkung}
\newtheorem*{beispiel}{Beispiel}

\begin{document}

\begin{titlepage}
\begin{center}
\textsc{\LARGE Lineare Algebra}\\[2.0cm]
\rule{\linewidth}{0.5mm}
Eine Mitschrift der Vorlesung Lineare Algebra I \& II von Prof. Dr. Scheithauer im WS/SS 14/15 an der TU Darmstadt.
\rule{\linewidth}{0.5mm}\\[2.0cm]
\begin{minipage}{0.4\textwidth}
\begin{flushleft}
\large \emph{Author:}\\\textsc{Johannes Struve}\\[1.0cm]
\large \emph{Koauthor:}\\\textsc{Daniel Kallendorf}
\end{flushleft}
\end{minipage}
\vfill
Zuletzt bearbeitet am {\large \today}.
\end{center}
\end{titlepage}

% Inhaltsverzeichnis
\tableofcontents

% nächste Seite
\newpage

\section{Grundbegriffe}
\subsection{Mengen und Abbildungen}
\subsubsection{Mengen}
Wir begnügen uns mit der naiven Definition einer \emph{Menge} als Zusammenfassung gewisser Objekte, der sog. Elemente dieser Menge. Sie ist eindeutig durch diese festgelegt. Einen präzisen Rahmen für die Mengenlehre liefert die \emph{ZFC-Axiomatik} (Zermelo \& Fraenkel). Es ist ein sicheres Verfahren, aus vorhandenen Mengen neue Mengen als Teilmengen zu beschreiben.
Wir definieren für Mengen $A$ und $B$:
\begin{itemize}
	\item \emph{Teilmenge} $A\subset B$
	\item \emph{Potenzmenge} $P(A)$
	\item \emph{Schnitt}- und \emph{Vereinigungsmenge} beliebiger Mengen $A\cap B$, $A\cup B$
	\item \emph{Mengendifferenz} $A\setminus B$
	\item \emph{karthesisches Produkt} $A\times B$
	\item \emph{Familie} $(a_i)_{i\in I}$
\end{itemize}
\newpage

\end{document}