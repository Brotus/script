% Lade tudreport
\documentclass[linedtoc,longdoc,colorback,accentcolor=tud1a,bigchapter,leqno,12pt,article]{tudreport}

% ----------------------------------------------
% Verwendete Pakete
% ----------------------------------------------

\usepackage[T1]{fontenc}
\usepackage[ngerman]{babel}
\usepackage[utf8]{inputenc}

% \usepackage{amsmath}
% \usepackage{amssymb}
% \usepackage{amstext}
\usepackage{amsthm}
% \usepackage{dsfont}
% \usepackage{booktabs}


% ----------------------------------------------
% Vortragsspezifische Definitionen
% ----------------------------------------------

%\theoremstyle{plain}
%\newtheorem{satz}{Satz}
%\newtheorem{korollar}{Korollar}

\theoremstyle{plain}
%\theoremstyle{definition}
\newtheorem*{definition}{Definition}
\newtheorem*{satz}{Satz}
\newtheorem*{theorem}{Theorem}

\theoremstyle{plain}
\newtheorem*{lemma}{Lemma}
\newtheorem*{korollar}{Korollar}

\theoremstyle{remark}
\newtheorem*{bemerkung}{Bemerkung}
\newtheorem*{beispiel}{Beispiel}

% Abkürzungen
\newcommand{\gdw}{\Leftrightarrow}
\newcommand{\N}{\mathbb{N}}
\newcommand{\Z}{\mathbb{Z}}
\newcommand{\Q}{\mathbb{Q}}
\newcommand{\R}{\mathbb{R}}
\newcommand{\C}{\mathbb{C}}
\newcommand{\F}{\mathbb{F}}
\newcommand{\impl}{\Rightarrow}
\newcommand{\la}{\lambda}
\newcommand{\al}{\alpha}
\newcommand{\SeiV}{Sei $V$ ein $K$-Vektorraum }
\newcommand{\summeA}{\sum_{i\in I}}
\newcommand{\summeB}{\sum_{i=1}^n}
\newcommand{\ol}[1]{\overline{#1}}
\newcommand{\norm}[1]{\|#1\|}
\newcommand{\fa}[1]{\mathop{\forall}\limits_{#1}}
\newcommand{\ex}[1]{\mathop{\exists}\limits_{#1}}
\newcommand{\sk}{\mbox{}\\*}

%rowcounter zurücksetzen
\newcommand{\resetRows}{\setcounter{equation}{0}}


% ----------------------------------------------
% Passe Ränder an (Korrekturrand für Abschlussarbeiten):
\geometry{tmargin=3cm,bmargin=3cm,lmargin=3cm,rmargin=3cm}
% ----------------------------------------------


% ----------------------------------------------
% Titelinformationen
% ----------------------------------------------
% Umlaute mit extra "\".

\title{Einführung in die Stochastik}
\author{Johannes Struve\\Timo Seiche\\Julian Keinrath}
\institution{Fachbereich Mathematik,\protect\\ Technische Universit\"at Darmstadt}
\date{heute}


% ----------------------------------------------
\begin{document}


% ----------------------------------------------
% Titel

\maketitle


% ----------------------------------------------
% Inhaltsverzeichnis

\tableofcontents
\newpage

% ----------------------------------------------
\chapter{Das mathematische Modell des Zufalls}
\textbf{Ziel:} Mathematische Modellierung \emph{zufälliger} Phänomene.\\
Ursachen für das Auftreten von Zufall:
\begin{itemize}
\item unvollständige Information
\item wurde zur Vereinfachung künstlich eingeführt
\end{itemize}
\section{Der Begriff der Wahrscheinlichkeit}
Ausgangspunkt: Zufallsexperiment (kurz: ZE)
\begin{definition}
Ein \emph{Zufallsexperiment} ist ein Experiment mit vorher unbestimmtem Ergebnis, das im Prinzip unbeeinflusst voneinander beliebig oft wiederholt werden kann.
\end{definition}
\begin{beispiel}
für ZE:\\
\begin{itemize}
\item Werfen eines Würfels, Ergebnis ist Zahl oben
\item Wiederholtes Werfen eines Würfels, das Ergebnis ist die Zahl der Würfe, bis zum ersten Mal $6$ gewürfelt wird.
\end{itemize}
kein ZE:
\begin{itemize}
\item Nächste Bundestagswahl (kann nicht unbeeinflusst voneinander wiederholt werden)
\end{itemize}
\end{beispiel}
Ziel: Aussagen über Ergebnisse von ZE:
\begin{definition}
\begin{itemize}
\item Die Menge aller möglichen Ergebnisse des ZE wird als \emph{Grundmenge} $\Omega$ bezeichnet.
\item Jede Teilmenge der Grundmenge heißt \emph{Ereignis}. Die einelementigen Teilmengen der Grundmenge heißen \emph{Elementarereignis}.
\item Ein Ereignis $A$ \emph{tritt ein}, falls das Ergebnis des Zufallsexperiments in $A$ liegt. Andernfalls tritt $A$ nicht ein.
\end{itemize}
\end{definition}
\begin{beispiel}
Werfen eines echten Würfels. Ergebnis = Zahl oben
\[\impl\Omega=\{ 1,2,3,4,5,6\}\]
Das Ereignis $A=\{ 2,4,6\}$ tritt genau dann ein, wenn eine gerade Zahl gewürfelt wird.
\end{beispiel}
\begin{definition}
ZE mit Grundmenge $\Omega$.
\end{definition}

\end{document}
