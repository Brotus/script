\documentclass[a4paper,12pt,fleqn]{scrartcl}
 
%**** Benutzte Packages

% utf8 Zeichen verwendbar
\usepackage[utf8x]{inputenc}
\usepackage[T1]{fontenc}
\usepackage[ngerman]{babel}
\usepackage{amsmath}
\usepackage{amssymb}
\usepackage{amsfonts}
\usepackage{amsthm}
\usepackage{tikz}
\usepackage{enumerate}
%http://www.guntherkrauss.de/computer/tex/diagramme.html
\usepackage[arrow, matrix, curve]{xy}


%**** Selbst definierte Befehle

% Abkürzungen

% Äquivalenzpfeil
\newcommand{\gdw}{\Leftrightarrow}
% natürliche Zahlen
\newcommand{\N}{\mathbb{N}}
% reelle Zahlen
\newcommand{\R}{\mathbb{R}}
% kaligraphische Buchstaben
\newcommand{\m}[1]{\mathcal{ #1 }}
% Für Wahrscheinlichkeit, \prob{A}=P(A)
\newcommand{\prob}[1]{\text{P(} #1 \text{)}}
\newcommand{\p}[1]{\text{P(} #1 \text{)}}
% Implikation
\newcommand{\impl}{\Rightarrow}
% geschlossenes reelles Einheitsintervall
\newcommand{\unit}{\left[ 0,1\right]}

\newcommand{\ol}[1]{\overline{#1}}
\newcommand{\fa}[1]{\mathop{\forall}\limits_{#1}}
\newcommand{\ex}[1]{\mathop{\exists}\limits_{#1}}
\newcommand{\sk}{\mbox{}\\*}

%rowcounter zurücksetzen
\newcommand{\resetRows}{\setcounter{equation}{0}}

%**** Theoreme

\theoremstyle{definition}
\newtheorem*{definition}{Definition}
\newtheorem*{satz}{Satz}
\newtheorem*{theorem}{Theorem}

\theoremstyle{plain}
\newtheorem*{lemma}{Lemma}
\newtheorem*{korollar}{Korollar}

\theoremstyle{remark}
\newtheorem*{bemerkung}{Bemerkung}
\newtheorem*{beispiel}{Beispiel}

\begin{document}

\begin{titlepage}
\begin{center}
\textsc{\LARGE Einführung in die Stochastik}\\[2.0cm]
\rule{\linewidth}{0.5mm}
Eine Mitschrift der Vorlesung Einführung in die Stochastik von Prof. Dr. Kohler im SS 16 an der TU Darmstadt.
\rule{\linewidth}{0.5mm}\\[2.0cm]
\begin{minipage}{0.4\textwidth}
\begin{flushleft}
\large \emph{Author:}\\\textsc{Johannes Struve\\Timo Seiche\\Julian Keinrath}\\[1.0cm]
\end{flushleft}
\end{minipage}
\vfill
Zuletzt bearbeitet am {\large \today}.
\end{center}
\end{titlepage}

% Inhaltsverzeichnis
\tableofcontents

% nächste Seite
\newpage

\section{Das mathematische Modell des Zufalls}
\textbf{Ziel:} Mathematische Modellierung \emph{zufälliger} Phänomene.\\
Ursachen für das Auftreten von Zufall:
\begin{itemize}
\item unvollständige Information
\item wurde zur Vereinfachung künstlich eingeführt
\end{itemize}
\subsection{Der Begriff der Wahrscheinlichkeit}
Ausgangspunkt: Zufallsexperiment (kurz: ZE)
\begin{definition}
Ein \emph{Zufallsexperiment} ist ein Experiment mit vorher unbestimmtem Ergebnis, das im Prinzip unbeeinflusst voneinander beliebig oft wiederholt werden kann.
\end{definition}
\begin{beispiel}
für ZE:
\begin{itemize}
\item Werfen eines Würfels, Ergebnis ist Zahl oben
\item Wiederholtes Werfen eines Würfels, das Ergebnis ist die Zahl der Würfe, bis zum ersten Mal $6$ gewürfelt wird.
\end{itemize}
kein ZE:
\begin{itemize}
\item Nächste Bundestagswahl (kann nicht unbeeinflusst voneinander wiederholt werden)
\end{itemize}
\end{beispiel}
Ziel: Aussagen über Ergebnisse von ZE:
\begin{definition}
\begin{itemize}
\item Die Menge aller möglichen Ergebnisse des ZE wird als \emph{Grundmenge} $\Omega$ bezeichnet.
\item Jede Teilmenge der Grundmenge heißt \emph{Ereignis}. Die einelementigen Teilmengen der Grundmenge heißen \emph{Elementarereignis}.
\item Ein Ereignis $A$ \emph{tritt ein}, falls das Ergebnis des Zufallsexperiments in $A$ liegt. Andernfalls tritt $A$ nicht ein.
\end{itemize}
\end{definition}
\begin{beispiel}
Werfen eines echten Würfels. Ergebnis = Zahl oben
\[\impl\Omega=\{ 1,2,3,4,5,6\}\]
Das Ereignis $A=\{ 2,4,6\}$ tritt genau dann ein, wenn eine gerade Zahl gewürfelt wird.
\end{beispiel}
\begin{definition}
ZE mit Grundmenge $\Omega$. Sind $x_1,\ldots,x_n\in\Omega$ die bei wiederholtem Durchführen des ZE auftretenden Werte, so heißen
\[\mid \{ 1\leq i\leq n:x_i\in A\}\mid\]
bzw.
\[\frac{\mid \{ 1\leq i\leq n:x_i\in A\}\mid}{n}\]
die \emph{absolute} bzw. \emph{relative Häufigkeit} des Eintretens von $A$.
\end{definition}
\begin{bemerkung}[Empirisches Gesetz der großen Zahlen]
Führt man ein ZE unbeeinflusst voneinander immer wieder durch, so nähert sich für große Anzahlen von Wiederholungen die relative Häufigkeit des Eintretens eines beliebigen Ereignisses $A$ einer Zahl
\[\prob{A} \in\unit\]
an. Wir bezeichnen $\prob{A}$ als \emph{Wahrscheinlichkeit} von $A$.
\end{bemerkung}
\begin{beispiel}
Werfen eines echten Würfels, Ergebnis = Zahl oben\\
Relative Häufigkeiten der Elementarereignisse nähern sich immer mehr $1/6$ an.
\[\prob{\{ 1\}}=\ldots=\prob{\{ 6\} }=\frac{1}{6}\]
\end{beispiel}
Ziel im Weiteren: Bestimmung von Wahrscheinlichkeiten ohne wiederholte Durchführung des ZE. Manchmal ist die \emph{Kombinatorik} hilfreich dabei.

\subsection{Grundlagen der Kombinatorik}
Ziehen von $k$ Elementen aus einer Menge mit Mächtigkeit $n$.\\
\\
4 Möglichkeiten:
\begin{itemize}
\item Ziehen mit bzw. ohne Zurücklegen
\item Ziehen mit bzw. ohne Beachtung der Reihenfolge
\end{itemize}
Sei $N$ die Anzahl der möglichen Stichproben. Dann gilt...
\begin{enumerate}[a)]
\item ...beim \emph{Ziehen mit Zurücklegen und mit Beachtung der Reihenfolge}:
\[N = n \cdot n \cdot \ldots \cdot n = n^{k}\]
\item ...beim \emph{Ziehen ohne Zurücklegen und mit Beachtung der Reihenfolge}:
\[N = n \cdot (n-1) \cdot \ldots \cdot (n-k+1) = \frac{n!}{(n-k)!}\]
\item ...beim \emph{Ziehen ohne Zurücklegen und ohne Bachtung der Reihenfolge}:\\
Ordnet man jede der Stichproben auf alle $k!$ möglichen Arten (vgl. b)), so erhält man alle Stichproben beim Ziehen ohne Zurücklegen und mit Beachtung der Reihenfolge.
\[\impl N \cdot k! = \frac{n!}{(n-k)!}\]
\[\impl N = \frac{n!}{(n-k)! \cdot k!} = \binom{n}{k}\]
\item ...beim \emph{Ziehen mit Zurücklegen und ohne Beachtung der Reihenfolge}:
\[N = \binom{n+k-1}{k}\]
Begründung: Gesucht ist $\mid \Omega \mid$, wobei 
\[\Omega=\{(x_1, \ldots, x_k) \in \N^{k}:1 \leq x_1 \leq x_2 \leq \ldots \leq x_k \leq n\}.\]
Setze \[\Omega^\prime=\{(y_1, \ldots, y_k) \in \N^{k}:1 \leq y_1 < y_2 < \ldots < y_k \leq n+k-1\}\] und definiere $f: \Omega \rightarrow \Omega^\prime$ durch $f((x_1, x_2, x_3, \ldots, x_k)) = (x_1, x_2+1, x_3+2, \ldots, x_k+k-1)$. Zu zeigen ist, dass $f$ eine (wohldefinierte) bijektive Abbildung ist. Dazu:
\begin{enumerate}[(i)]
\item $f$ ist wohldefiniert, denn $1 \leq x_1 \leq \ldots \leq x_k \impl 1 \leq x_1 < \ldots < x_k+k-1$.
\item $f$ ist injektiv:
\begin{align*}
&f((x_1, x_2, \ldots, x_k)) = f((\widetilde{x_1}, \widetilde{x_2}, \ldots, \widetilde{x_k})) \\
\impl & (x_1, x_2+1, \ldots, x_k+k-1) = (\widetilde{x_1}, \widetilde{x_2}+1, \ldots, \widetilde{x_k}+k-1) \\
\impl & (x_1, x_2, \ldots, x_k) = (\widetilde{x_1}, \widetilde{x_2}, \ldots, \widetilde{x_k})
\end{align*}
\item $f$ ist surjektiv:
\begin{align*}
&(y_1, y_2, \ldots, y_k) \in \Omega^\prime \\
\impl & 1 \leq y_1 < y_2 <\ldots < y_k \leq n+k-1 \\
\impl & 1 \leq y_1 \leq y_2-1 \leq \ldots \leq y_k-k+1 \leq n \\
\impl & (y_1,y_2-1, \ldots , y_k-k+1) \in \Omega
\end{align*}
Dann ist $\widetilde{y} \in \Omega$ und $f(\widetilde{y})=(y_1, \ldots, y_k) \in \Omega^\prime$. \\
$\impl \mid \Omega \mid = \mid \Omega^\prime \mid = \binom{n+k-1}{k}$
\end{enumerate}
\end{enumerate}
\begin{beispiel}[Binomischer Lehrsatz]
Für $a, b \in \R, n \in \N$ gilt 
\[(a+b)^{n} = \sum_{k=0}^{n} \binom{n}{k} a^{k} b^{n-k}.\]
Begründung:
\[(a+b)^{n} = \sum_{k=0}^{n} n_k a^{k} b^{n-k}\]
Hier bezeichnet $n_k$ die Anzahl der Mögichkeiten, um aus $n$ Faktoren $k$ auszuwählen (bei denen $a$ gewählt wird) ohne Zurücklegen und ohne Beachtung der Reihenfolge. Ohne Zurücklegen, da kein Faktor doppelt gewählt werden kann; ohne Beachtung der Reihenfolge, da Reihenfolge beim Produkt egal ist. Aus c) folgt $n_k = \binom{n}{k}$.
\end{beispiel}

\subsection{Drei Beispiele}
\begin{beispiel}
Ein (echter) Würfel wird solange geworfen, bis er zum ersten Mal mit $6$ oben landet. Wie groß ist die Wahrscheinlichkeit, dass die Anzahl der Würfe gerade ist?\\
Das Ergebnis des ZE ist die Zahl der Würfe bis zur ersten $6$, wobei der letzte Wurf mitgezählt wird.\\
Hier ist
\[\Omega = \{1,2,3,\ldots\} \cup \{\infty\}\]
und gesucht ist die Wahrscheinlichkeit des Ereignisses
\[A=\{2,4,6,8,\ldots\}.\]
Ansatz: $\prob{A} = \sum_{k \in A} \prob{\{k\}}$ \\
Die Wahrscheinlichkeit eines Ereignisses ist die Summe der Wahrscheinlichkeiten aller darin enthaltenen Elementarereignisse.\\
\\
Bestimmung von $\prob{\{k\}}$ für $k \in \N$: \\
Stimmt mit Wahrscheinlichkeit überein, dass beim k-maligen Werfen eines echten Würfels (Ziehen von $k$ Elementen aus einer Grundmenge von Umfang $6$ mit Zurücklegen, mit Beachtung der Reihenfolge)\\
Von diesen erscheint bei genau
\[5^{k-1} \cdot 1 = 5^{k-1}\]
beim $k$-ten Wurf die erste $6$. Jedes der $6^{k}$ Elementarereignisse hat (wegen der Symmetrie) die gleiche Wahrscheinlichkeit von $6^{-k}$.
\[\impl \prob{\{k\}}=\frac{5^{k-1}}{6^{k}}=\frac{1}{6}(\frac{5}{6})^{k-1}\]
Damit ist
\begin{align*}
\prob{A}&=\sum_{k \in A} \prob{\{k\}}=\frac{1}{6}\sum_{k \in A}(\frac{5}{6})^{k-1}=\frac{1}{6}\sum_{i=1}^{\infty}(\frac{5}{6})^{2i-1}=\frac{1}{6}\sum_{i=1}^{\infty}(\frac{5}{6})^{2(i-1)+1} \\
&=\frac{1}{6}\cdot\frac{5}{6}\sum_{i=0}^{\infty}((\frac{5}{6})^{2})^{i}=\frac{5}{36}\sum_{i=1}^{\infty}(\frac{25}{36})^{i}=\frac{5}{36}\cdot\frac{1}{1-\frac{25}{36}}=\frac{5}{36-25} \\
&=\frac{5}{11}.
\end{align*}
\end{beispiel}
\begin{beispiel}
Jackpot beim Lotto \glqq6 aus 49\grqq im Dezember 2007: 43 Mio. Euro \\
Beobachtung: In den $4599$ Ziehungen von Oktober 1955 bis Dezember 2007 wurde die $38$ am häufigsten gezogen, nämlich $614$-mal. \\
Zum Vergleich
\[\frac{4599 \cdot 6}{49} = 563\]
Frage: Zufall? \\
\underline{Grundidee in der Stochastik zur Beantwortung dieser Frage:}
\begin{itemize}
\item Nimm an, dass die $6$ Zahlen zufällig gezogen werden, d.h. jede Zahlenkombination tritt mit gleicher Wahrscheinlichkeit auf.
\item Berechne unter dieser Annahme Wahrscheinlichkeit, dass das Resultat beobachtet wird, welches so stark gegen die Annahme spricht, wie das tatsächlich beobachtete.
\item Falls Wahrscheinlichkeit klein ist (z.B. $\leq 0.05$), verwirf die Annahme, andernfalls nicht.
\end{itemize}
Im Folgenden: Bestimmen der Wahrscheinlichkeit, dass bei $n = 4599$ Ziehungen die $38$ mindestens $614$-mal gezogen wird. \\
Dazu:
\begin{align*}
&\prob{\text{Bei } n \text{ Ziehungen } 38 \text{ mindestens } 614 \text{-mal}}\\
= &\sum_{k = 614}^{n} (\prob{\text{Bei } n \text{ Ziehungen } 38 \text{ genau } k \text{-mal}})
\end{align*}
Nun gilt: \\
Wahrscheinlichkeit, dass $38$ bei einer Ziehung gezogen wird
\begin{align*}
&= \frac{ \text{\# günstigste Fälle}}{\text{\# mögliche Fälle}} \\
&= \frac{1 \cdot \binom{48}{5}}{\binom{49}{6}} \\
&= \frac{\frac{48!}{(48-5)! \cdot 5!}}{\frac{49!}{(49-6)!}} = \frac{6}{49} =: p
\end{align*}
(Obere Vorgehensweise benutzt ohne Zurücklegen und ohne Beachtung der Reihenfolge, in diesem Fall ist es aber auch möglich mit Beachtung der Reihenfolge zu betrachten.) \\

Wahrscheinlichkeit, dass $38$ bei $n = 4599$ aufeinanderfolgenden Ziehungen genau $k$-mal auftritt.
\begin{align*}
&= \frac{\text{\# günstige Fälle}}{\text{\# mögliche Fälle}} \\
&= \frac{\binom{n}{k} \cdot \binom{48}{5}^k \cdot (\binom{49}{6} - \binom{48}{5})^{(n-k)} }{\binom{49}{6}^n}
\end{align*}
Anmerkungen zu den Faktoren: \\ 
$\binom{n}{k}$ = Position der $k$ Ziehungen \\
$\binom{48}{5}^k$ = $k$ Ziehungen mit $38$ \newpage
\begin{align*}
(\binom{49}{6} - \binom{48}{5})^{(n-k)} &= n-k \text{ Ziehungen ohne } 38 \\
&= \binom{n}{k} \cdot (\frac{\binom{48}{5}}{\binom{49}{6}})^k \cdot (1 - \frac{\binom{48}{5}}{\binom{49}{6}})^{(n-k)} \\
&= \binom{n}{k} \cdot p^k \cdot (1-p)^{(n-k)} \\
\impl &\prob{\text{Bei } n \text{ Ziehungen } 38 \text{ mindestens } 614\text{-mal}} \\
&=\sum_{k=614}^n (\binom{n}{k} \cdot p^k \cdot (1-p)^{(n-k)}) \approx 0.01
\end{align*}
Dies ist sehr klein. Aber es spricht eigentlich jedes Resultat gegen die Annahme, bei dem irgendeine Zahl mindesten $614$-mal auftritt. \\
Wahrscheinlichkeit, dass irgendeine der $49$ Zahlen bei $n=4599$ Ziehungen mindestens $614$-mal gezogen wird (per Computersimulation) 
\[\approx 0.47,\]
was nicht klein ist! \\
Fazit: Annahme kann nicht verworfen werden.
\end{beispiel}

% Ab hier 9.5.
\begin{beispiel}
Eine Zahl wird rein zufällig aus $\left[ 0,5\right]$ gezogen. wie groß ist die Wahrscheinlichkeit, dass sie zwischen $2$ und $4$ liegt.  Hier ist $\Omega=\left[ 0,5\right]$, gesucht ist die Wahrscheinlichkeit von $A=\left[ 2,4\right]$. Da alle Teile von $\Omega$ "gleichberechtigt" sind, machen wir den Ansatz
\[\prob{\left[ a,b\right]}.\]
Es folgt
\[\prob{\left[ 2,4\right]}=\frac{4-2}{5}=0.4.\]
\begin{bemerkung}
Hier ist
\[\prob{A}=\sum_{\omega\in A}\prob{\{\omega \} }\]
nicht möglich, da $\prob{\{ x\} }$ für alle $x\in\Omega$.
\end{bemerkung}
\end{beispiel}
\begin{bemerkung}[Problem]
Ansätze unsystematisch! Gibt es auch systematischen Zugang?
\end{bemerkung}

\subsection{Der Begriff des Wahrscheinlichkeitsraumes}
\begin{definition}
$\Omega\neq\emptyset$. Eine Menge $\m{A}\subset\m{P}(\Omega)$  heißt $\sigma$-Algebra (über $\Omega$), falls gilt
\begin{enumerate}
\item $\emptyset\in\m{A}$
\item $A\in\m{A}\impl \Omega\setminus A\in\m{A}$
\item $(A_n)_{n\in\N}\impl\bigcup_{n\in\N}A_n$
\end{enumerate}
\end{definition}
\begin{beispiel}
\begin{enumerate}[a]
\item $\m{A}=\m{P}(\Omega)$
\item $\m{A}=\{\emptyset,\Omega\}$
\item Die kleinste $\sigma$-Algebra über $\R$, die alle Intervalle enthält, ist die sogenannte \emph{Borelsche $\sigma$-Algebra} $\m{B}$
\end{enumerate}
\end{beispiel}
\begin{bemerkung}[Zur Existenz von $\m{B}$]
$I\neq\emptyset$ Indexmenge, $\m{A}_i$ $\sigma$-Algebra über $\Omega$ für $i\in I$ impliziert $\cap_{i\in I}\m{A}_i$ ist $\sigma$-Algebra über $\Omega$. Setze dann:
\[\m{B}:=\bigcap\{\m{A}|\m{A}\text{ ist }\sigma\text{-Algebra über }\R\text{ und enthält alle Intervalle}\}\]
$\m{B}$ ist nicht gleich der Potenzmenge von $\R$.
\end{bemerkung}
\begin{definition}
$\Omega\neq\emptyset,\m{A} \sigma$-Algebra über $\Omega$. Jede Abbildung $\text{P}:\m{A}\to\R$ mit
\begin{enumerate}[i]
\item $\fa{A\in\m{A}}:\prob{A}\in\left[ 0,1\right]$,
\item $\prob{\emptyset}=0,\prob{\Omega}=1$,
\item $A\in\m{A}\impl \prob{A^C}=1-\prob{A}$,
\item $A,B\in\m{A},A\subset B\impl \prob{A}\leq \prob{B}$,
\item $(A_i)_{i\in I}$, $I$ abzählbar, $A_i$ paarweise disjunkt: $\prob{\bigcup_{i\in I}} A_i=\sum_{i\in I}\prob{A_k}$
\end{enumerate}
heißt \emph{Wahrscheinlichkeitsmaß} (kurz: W-Maß).
\end{definition}
\begin{lemma}
$\text{P}:\m{A}\to\R$ ist W-Maß, genau dann wenn:
\begin{enumerate}
\item $\p{A}\geq 0$
\item $\p{\Omega}=1$
\item $\sigma$-Additivität
\end{enumerate}
\end{lemma}
\begin{proof}
"$\impl$" trivial. "$\Leftarrow$": Es gilt
\begin{itemize}
\item $\p{\emptyset}=\p{\bigcup_{n\in\N}\emptyset}=\sum_{n\in\N}\p{\emptyset}\impl\p{\emptyset}=0$
\item $\p{ A\cup B}=\p{ A\cup B\cup\emptyset\ldots}=\p{ A}+\p{ B}$
\item $A\subseteq B\impl B=A\cup (B\setminus A)\impl\p{ B}=\p{A}+\p{B\setminus A}\geq\p{A}$
\item $1=\p{\Omega}=\p{A}+\p{\Omega\setminus A}\impl\p{A^C}=1-\p{A}$
\end{itemize}
\end{proof}
\begin{definition}
Ist $\Omega\neq\emptyset$, $\m{A}$ $\sigma$-Algebra über $\Omega$ und $\text{P}:\m{A}\to\R$ ein W-Maß, so heißt $(\Omega,\m{A},\text{P})$ \emph{Wahrscheinlichkeitsraum} (kurz: W-Raum). In diesem Fall heißt $\p{A}$ \emph{Wahrscheinlichkeit} (kurz: Wk.).
\end{definition}
\end{document}
