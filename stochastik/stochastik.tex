\documentclass[a4paper,12pt,fleqn]{scrartcl}
 
%**** Benutzte Packages

% utf8 Zeichen verwendbar
\usepackage[utf8x]{inputenc}
\usepackage[T1]{fontenc}
\usepackage[ngerman]{babel}
\usepackage{amsmath}
\usepackage{amssymb}
\usepackage{amsfonts}
\usepackage{amsthm}
\usepackage{tikz}
\usepackage{mathtools}
%http://www.guntherkrauss.de/computer/tex/diagramme.html
\usepackage[arrow, matrix, curve]{xy}


%**** Selbst definierte Befehle

% Abkürzungen

% Äquivalenzpfeil
\newcommand{\gdw}{\Leftrightarrow}
% natürliche Zahlen
\newcommand{\N}{\mathbb{N}}
% reelle Zahlen
\newcommand{\R}{\mathbb{R}}
% Für Wahrscheinlichkeit, \prob{A}=P(A)
\newcommand{\prob}[1]{\text{P(} #1 \text{)}}
% Implikation
\newcommand{\impl}{\Rightarrow}
% geschlossenes reelles Einheitsintervall
\newcommand{\unit}{\left[ 0,1\right]}

\newcommand{\ol}[1]{\overline{#1}}
\newcommand{\fa}[1]{\mathop{\forall}\limits_{#1}}
\newcommand{\ex}[1]{\mathop{\exists}\limits_{#1}}
\newcommand{\sk}{\mbox{}\\*}

%rowcounter zurücksetzen
\newcommand{\resetRows}{\setcounter{equation}{0}}

%**** Theoreme

\theoremstyle{definition}
\newtheorem*{definition}{Definition}
\newtheorem*{satz}{Satz}
\newtheorem*{theorem}{Theorem}

\theoremstyle{plain}
\newtheorem*{lemma}{Lemma}
\newtheorem*{korollar}{Korollar}

\theoremstyle{remark}
\newtheorem*{bemerkung}{Bemerkung}
\newtheorem*{beispiel}{Beispiel}

\begin{document}

\begin{titlepage}
\begin{center}
\textsc{\LARGE Einführung in die Stochastik}\\[2.0cm]
\rule{\linewidth}{0.5mm}
Eine Mitschrift der Vorlesung Einführung in die Stochastik von Prof. Dr. Kohler im SS 16 an der TU Darmstadt.
\rule{\linewidth}{0.5mm}\\[2.0cm]
\begin{minipage}{0.4\textwidth}
\begin{flushleft}
\large \emph{Author:}\\\textsc{Johannes Struve\\Timo Seiche\\Julian Keinrath}\\[1.0cm]
\end{flushleft}
\end{minipage}
\vfill
Zuletzt bearbeitet am {\large \today}.
\end{center}
\end{titlepage}

% Inhaltsverzeichnis
\tableofcontents

% nächste Seite
\newpage

\section{Das mathematische Modell des Zufalls}
\textbf{Ziel:} Mathematische Modellierung \emph{zufälliger} Phänomene.\\
Ursachen für das Auftreten von Zufall:
\begin{itemize}
\item unvollständige Information
\item wurde zur Vereinfachung künstlich eingeführt
\end{itemize}
\subsection{Der Begriff der Wahrscheinlichkeit}
Ausgangspunkt: Zufallsexperiment (kurz: ZE)
\begin{definition}
Ein \emph{Zufallsexperiment} ist ein Experiment mit vorher unbestimmtem Ergebnis, das im Prinzip unbeeinflusst voneinander beliebig oft wiederholt werden kann.
\end{definition}
\begin{beispiel}
für ZE:\\
\begin{itemize}
\item Werfen eines Würfels, Ergebnis ist Zahl oben
\item Wiederholtes Werfen eines Würfels, das Ergebnis ist die Zahl der Würfe, bis zum ersten Mal $6$ gewürfelt wird.
\end{itemize}
kein ZE:
\begin{itemize}
\item Nächste Bundestagswahl (kann nicht unbeeinflusst voneinander wiederholt werden)
\end{itemize}
\end{beispiel}
Ziel: Aussagen über Ergebnisse von ZE:
\begin{definition}
\begin{itemize}
\item Die Menge aller möglichen Ergebnisse des ZE wird als \emph{Grundmenge} $\Omega$ bezeichnet.
\item Jede Teilmenge der Grundmenge heißt \emph{Ereignis}. Die einelementigen Teilmengen der Grundmenge heißen \emph{Elementarereignis}.
\item Ein Ereignis $A$ \emph{tritt ein}, falls das Ergebnis des Zufallsexperiments in $A$ liegt. Andernfalls tritt $A$ nicht ein.
\end{itemize}
\end{definition}
\begin{beispiel}
Werfen eines echten Würfels. Ergebnis = Zahl oben
\[\impl\Omega=\{ 1,2,3,4,5,6\}\]
Das Ereignis $A=\{ 2,4,6\}$ tritt genau dann ein, wenn eine gerade Zahl gewürfelt wird.
\end{beispiel}
\begin{definition}
ZE mit Grundmenge $\Omega$. Sind $x_1,\ldots,x_n\in\Omega$ die bei wiederholtem Durchführen des ZE auftretenden Werte, so heißen
\[\mid \{ 1\leq i\leq n:x_i\in A\}\mid\]
bzw.
\[\frac{\mid \{ 1\leq i\leq n:x_i\in A\}\mid}{n}\]
die \emph{absolute} bzw. \emph{relative Häufigkeit} des Eintretens von $A$.
\end{definition}
\begin{bemerkung}[Empirisches Gesetz der großen Zahlen]
Führt man ein ZE unbeeinflusst voneinander immer wieder durch, so nähert sich für große Anzahlen von WIederholungen die relative Häufigkeit des Eintretens eines beliebigen Ereignisses $A$ einer Zahl
\[\prob{A} \in\unit\]
an. Wir bezeichnen $\prob{A}$ als \emph{Wahrscheinlichkeit} von $A$.
\end{bemerkung}
\begin{beispiel}
Werten eines echten Würfels, Ergebnis = Zahl oben\\
Relative Häufigkeiten der Elementarereignisse nähern sich immer mehr $1/6$ an.
\[\prob{\{ 1\}}=\ldots=\prob{\{ 6\} }=\frac{1}{6}\]
\end{beispiel}
Ziel im Weiteren: Bestimmung von Wahrscheinlichkeiten ohne wiederholte Durchführung des ZE. Manchmal ist die \emph{Kombinatorik} hilfreich dabei.
\end{document}
